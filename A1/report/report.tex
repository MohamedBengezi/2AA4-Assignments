\documentclass[12pt]{article}

\usepackage{graphicx}
\usepackage{paralist}
\usepackage{listings}
\usepackage{hyperref}
\hypersetup{colorlinks=true,
    linkcolor=blue,
    citecolor=blue,
    filecolor=blue,
    urlcolor=blue,
    unicode=false}

\oddsidemargin 0mm
\evensidemargin 0mm
\textwidth 160mm
\textheight 200mm

\pagestyle {plain}
\pagenumbering{arabic}

\newcounter{stepnum}

\title{Assignment 1 Solution}
\author{Mohamed Bengezi bengezim}

\begin {document}

\maketitle

The purpose of this software design exercise is to write a Python program that
creates, uses, and tests an ADT that stores circles.  The program consist of the
following files: {\tt CircleADT.py}, {\tt Statistics.py}, {\tt testCircles.py}
and {\tt Makefile}, as shown in Appendix



\section*{Testing of the Original Program}

The results of testing my files, specifically CircleADT.py and Statistics.py were all successful.
 Most of the simple functions had simple test cases, but more complicated ones, such as intersect(), or rank()
 had specialized ones. The following were specific test cases that were notable:
\begin{itemize}
\item In intersect(), I had the test case of two circles with the same centre, but different radii.
This test case was expected to return True, seeing as they are fully intersecting. It
successfully did so.
\item Another intersect() test case that is notable is where the edges of the circles are just touching. 
With my assumptions, I expected the result to be false, seeing as they have not actually intersected one
another. The actual result matched the expected
\item In rank, I made the assumption that if there were two circles in the list with the same radius, the 
first one would be have a higher rank than the second one

\end{itemize}


\begin{figure}
\centering
\includegraphics[width = 0.8\textwidth]{C:/Users/Mohamed/Desktop/img1.PNG}
\caption{Results of testing My files}
\label{Figure : example}
\end{figure}


\newpage

\section*{Testing of the Partner Files}
As seen in the next figure, my partner's files only failed the two test cases mentioned above.
The main reason for this is that they have made different assumptions than me:
\begin{itemize}
\item One assumption is that when the circles are just touching, they are intersecting. This
is a reasonable assumption.
\item Another is that when the circles are fully overlapped, they are not intersecting. I believe
this is not a reasonable assumption, seeing as mathematically they are intersecting in this case.
\end{itemize}

\begin{figure}
\centering
\includegraphics[width = 0.8\textwidth]{C:/Users/Mohamed/Desktop/img4.jpg}
\caption{Results of testing Partner files}
\label{Figure : example}
\end{figure}

\newpage
\section*{Discussion}
\subsection{Issues with my code}
As discussed earlier, the only controversial test cases were from intersect() and rank(). In summary, all my functions 
behaved expectedly. At first, I had tested my insideBox() method and found that it wasn't behaving expectedly. This 
was due to a wrong implementation of the if statement conditions. Once that was corrected, the functions all proceeded
as expected


\subsection{Issues with partners code}
When it came to testing my partner's files, all functions behaved expectedly, except for inersect(). This was due to 
different assumptions made by me and my partner. This shows me that when working on projects with others, or even
solely, you should take into consideration the assumptions that others might make, and how it could affect the correctness
of your code.


\subsection{Module Implementation and Specification}
As for specification of the module's, the specification and implementation of the function were all very similar, with the exception 
of intersect(). This would explain why their intersect method failed my test cases. I directly used the distance between points  formula and compared it with the radii, while they more so used thier own logic.


\subsection{Handling  $\pi$}
I handled the value of  $\pi$ by using the math library in python. The reasoning for this is that seeing as it is a default python
library, the constant is widely used, and accurate enough for this situation (11 decimal places). Seeing as it is not literally defined in any of my functions, 
it is defined explicitly, in the math library. The scope of  $\pi$ is then global, because it is used everywhere. 
\newpage

\lstset{language=Python, basicstyle=\tiny,breaklines=true,showspaces=false,showstringspaces=false,breakatwhitespace=true}

\def\thesection{\Alph{section}} 

\section{Code for CircleADT.py} \label{CircleSect}

\noindent \lstinputlisting{C:/Users/Mohamed/bengezim/A1/src/CircleADT.py}

\newpage



\section{Code for Statistics.py}

\noindent \lstinputlisting{C:/Users/Mohamed/bengezim/A1/src/Statistics.py}


\newpage

\section{Code for testCircles.py} \label{testSect}

\noindent \lstinputlisting{C:/Users/Mohamed/bengezim/A1/src/testCircles.py}


\newpage

\section{Partner's CircleADT.py} \label{PartSec}

\noindent \lstinputlisting{C:/Users/Mohamed/Desktop/CircleADT.py}


\newpage

\section{Partner's Statistics.py} \label{PartSec1}

\noindent \lstinputlisting{C:/Users/Mohamed/Desktop/Statistics.py}
    
\newpage


\section{Makefile} \label{MakefileSect}
%\lstset{language=make}
%\noindent \lstinputlisting{C:/Users/Mohamed/bengezim/A1/Makefile}
Figure is on next page....

\begin{figure}[b]


\centering

\includegraphics[width = \textwidth]{C:/Users/Mohamed/Desktop/img.PNG}

\caption{Makefile}

\label{Figure : example}

\end{figure}



\end {document}
