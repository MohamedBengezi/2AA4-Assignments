\documentclass[12pt]{article}

\usepackage{graphicx}
\usepackage{paralist}
\usepackage{amsfonts}
\usepackage{listings}
\usepackage{hyperref}
\usepackage{indentfirst}
\usepackage[T1]{fontenc}
\usepackage{fullpage}
\usepackage{titling}
\usepackage{booktabs}
\usepackage{enumitem}
\usepackage{listings}
\usepackage{color}
\usepackage[dvipsnames]{xcolor}
\usepackage{tabularx,ragged2e}
\hypersetup{colorlinks=true,
    linkcolor=blue,
    citecolor=blue,
    filecolor=blue,
    urlcolor=blue,
    unicode=false}

\oddsidemargin 0mm
\evensidemargin 0mm
\textwidth 160mm
\textheight 200mm

\pagestyle {plain}
\pagenumbering{arabic}

\newcounter{stepnum}

\title{Assignment 4 - 2AA4}
\author{Mohamed Bengezi bengezim}

\begin {document}

\maketitle
The contents of this report include the specifications of the game state for a game a Battleship.

\newpage

\section* {Battleship Module}

\subsection* {Module}

Battleship

\subsection* {Uses}

PointT, ShipT, GameStateT

\subsection* {Syntax}

\subsubsection* {Exported Access Programs}

\begin{tabular}{| l | l | l | p{4cm} |}
\hline
\textbf{Routine name} & \textbf{In} & \textbf{Out} & \textbf{Exceptions}\\
\hline
init & sequence[5] of ShipT&  & \\
  & sequence[5] of ShipT &  & \\
\hline
all\_shots & & sequence of PointT & \\
\hline
place\_shot & PointT & boolean & InvalidShotException\\
\hline
is\_winner &  & boolean & \\
\hline


\end{tabular}

\subsection* {Semantics}

\subsubsection* {State Variables}

\noindent $shots\_taken$: sequence of PointT\\
$\mathit{firstPlayerTurn}$: boolean \\
$first\_player$: GameStateT for the first player \\
$second\_player$: GameStateT for the second player \\

\subsubsection* {State Invariant}

None

\subsubsection* {Assumptions}

The init method is called for the abstract object before any other access routine is called for that
object.  The init method can be used to return the state of the game to the state of a new game.

\subsubsection* {Access Routine Semantics}

\noindent init($\mathit{firstPlayerShipList}, \mathit{secondPlayerShipList}$):
\begin{itemize}
\item transition: 
$\mathit{shots\_taken}, \mathit{firstPlayerTurn}, \mathit{first\_player}, \mathit{second\_player} := <>,  true,\\
 \mbox{new GameStateT(firstPlayerShipList)}, \mbox{new GameStateT(secondPlayerShipList)} $
\item exception: None
\end{itemize}


\noindent all\_shots():
\begin{itemize}
\item output: $out := shots\_taken$
\item exception: None
\end{itemize}

\noindent place\_shot($p$):
\begin{itemize}
\item transition-output: $firstPlayerTurn, out := \lnot firstPlayerTurn$ and $shots\_taken$ such that $shots\_taken = pre(shots\_taken)[0..|pre(shots\_taken)|-1]||\mathit{<}p\mathit{>}, (firstPlayerTurn \Rightarrow second\_player.has\_been\_shot(p) | \lnot firstPlayerTurn \Rightarrow first\_player.has\_been\_shot(p))$
\item exception: $exc := (firstPlayerTurn \Rightarrow \exists( i : \mathbb{I} | (i \% 2 = 0) \land (0 \leq i < |shots\_taken|) : repeatedShot(p, shots\_taken[i]))| \lnot firstPlayerTurn \Rightarrow \exists( i : \mathbb{I} | (i \% 2 = 1) \land (1 \leq i < |shots\_taken|) : repeatedShot(p, shots\_taken[i]))) \Rightarrow$ InvalidShotException
\end{itemize}

\noindent is\_winner():
\begin{itemize}
\item output: $out := (firstPlayerTurn \Rightarrow second\_player.\mbox{all\_ships\_lost()}| \lnot firstPlayerTurn \Rightarrow first\_player.all\_ships\_lost())$
\item exception: None
\end{itemize}

\subsubsection* {Local Functions}
\textbf{repeatedShot} : PointT $\times$ PointT $\rightarrow$ boolean\\
~\newline
repeatedShot($p,q$) $\equiv (p.xcrd() = q.xcrd()) \land (p.ycrd() = q.ycrd())$

~\newline

\section* {Game State Module}

\subsection* {Template Module}

GameStateT

\subsection* {Uses}

ShipT, PointT

\subsection* {Syntax}

\subsubsection* {Exported Types}

GameStateT = ?

\subsubsection* {Exported Constants }
\noindent MAX\_ROWS =  10 
\newline
MAX\_COLUMNS = 10


\subsubsection* {Exported Access Programs}

\begin{tabular}{| l | l | l | p{6cm} |}
\hline
\textbf{Routine name} & \textbf{In} & \textbf{Out} & \textbf{Exceptions}\\
\hline
GameStateT & sequence of ShipT  & GameStateT & InvalidConfigurationException\\
\hline
has\_been\_shot & PointT & boolean & \\
\hline
all\_ships\_lost &  & boolean & \\
\hline
\end{tabular}

\subsection* {Semantics}

\subsubsection* {State Variables}

\noindent $\mathit{hitList}$: sequence <> of integer {\it //integer array representing the list of shots hit}\\
$\mathit{shipList}$: sequence <> of ShipT {\it //ShipT array representing the shipList}\\
$\mathit{hitIndex}$: Integer 

\subsubsection* {State Invariant}

None

\subsubsection* {Assumptions}

The GameStateT() constructor is called for each abstract object before any other access routine is called for that
object.  The constructor can only be called once.

\subsubsection* {Access Routine Semantics}

\noindent GameStateT($shipList$):
\begin{itemize}
\item transition: $\mathit{shipList}, \mathit{hitList}:= shipList, <>$
\item output: $out := \mathit{self}$
\item exception: 
\begin{eqnarray*}
\lefteqn{exc :=
 (\lnot (|shipList| = 5) \lor \lnot (shipList[0].get\_length() = 2)  \lor \lnot (shipList[1].get\_length() = 3) \lor} \\
& & \lnot (shipList[2].get\_length() = 3)  \lor \lnot (shipList[3].get\_length() = 4)  \lor \\
& &  \lnot (shipList[4].get\_length() = 5)  \lor\\
& &  (\exists( i : \mathbb{I} | 0 \leq i < |shipList| : \exists( j : \mathbb{I} | (0 \leq j < |shipList|) \land (i \neq j) :\\
& &  \mbox{conflict}(shipList[i], shipList[j]))))) \Rightarrow \mathrm{InvalidConfigurationException}
\end{eqnarray*}
\end{itemize}

\noindent has\_been\_shot($p$):
\begin{itemize}
\item transition-output: $hitList, out := (checkShot(shipList, p) \Rightarrow hitList[0..hitIndex-1] || < pre(hitList)[hitIndex] + 1>|| hitList[hitIndex + 1 .. |hitList| -1 ] where \\
 pointInLine(p, shipList[i].get\_head(), shipList[i].get\_tail()) = true |  \lnot checkShot(shipList, p) \Rightarrow hitList)checkShot(shipList, p)$
\item exception: None
\end{itemize}

\noindent all\_ships\_lost():
\begin{itemize}
\item output: $out := \forall( i : \mathbb{I} | 0 \leq i < |shipList| : shipList[i].\mbox{get\_length}() = hitList[i])$
\item exception: None
\end{itemize}


\subsubsection* {Local Functions}

\textbf{pointInLine} : PointT $\times$ PointT $\times$ PointT $\rightarrow$ boolean\\
~\newline
pointInLine$(p, head, tail) \equiv (head.dist(p) + tail.dist(p) = head.dist(tail))$

~\newline

\noindent \textbf{conflict} : ShipT $\times$ ShipT $\rightarrow$ boolean\\
~\newline

conflict$(first, second) \equiv \exists( i :PointT | pointInLine(i, first.get\_head(), firstget\_tail()) :pointInLine(i, second.get\_head(), second.get\_tail()) \rightarrow hitIndex = index_of_i )$

~\newline

\noindent \textbf{checkShot} : sequence of ShipT $\times$ PointT $\rightarrow$ boolean\\
~\newline
checkShot$(shipList, p) \equiv \exists( s : ShipT | s \in shipList : pointInLine(p, s.get\_head(), s.get\_tail()))$

~\newline

\section* {Ship Module}

\subsection* {Template Module}

ShipT

\subsection* {Uses}

PointT

\subsection* {Syntax}

\subsubsection* {Exported Types}

ShipT = ?

\subsubsection* {Exported Constants }

MAX\_SIZE =  5 {\it //Maximum size of a ship}\\
\newline
MIN\_SIZE = 2 {\it //Minimum size of a ship}\\


\subsubsection* {Exported Access Programs}

\begin{tabular}{| l | l | l | l |}
\hline
\textbf{Routine name} & \textbf{In} & \textbf{Out} & \textbf{Exceptions}\\
\hline
ShipT & integer, PointT, PointT & ShipT & InvalidShipException \\
\hline
get\_length &   &  integer  & \\
\hline 
get\_head & & PointT &  \\
\hline
get\_tail &  &  PointT & \\
\hline
\end{tabular}

\subsection* {Semantics}

\subsubsection* {State Variables}

$\mathit{head}$: PointT {\it //Start point of the boat}\\
$\mathit{tail}$: PointT {\it //End point of the boat}\\
$\mathit{length}$: integer {\it //Length of the boat}

\subsubsection* {State Invariant}


\subsubsection* {Assumptions}
The ShipT constructor is called for each abstract object before any other access routine is called for that
object.  The constructor can only be called once.

\subsubsection* {Access Routine Semantics}

\noindent ShipT($l, a, b$):
\begin{itemize}
\item transition: $\mathit{length}, \mathit{head}, \mathit{tail} := l, a, b$
\item output: $out := \mathit{self}$
\item exception: $exc := (\lnot (MIN\_SIZE \leq length \leq \mbox{MAX\_SIZE}) \vee ((head.xcrd() \neq tail.xrd()) \land (head.ycrd() \neq tail.ycrd())) \Rightarrow
\mbox{InvalidShipException})$
\end{itemize}

\noindent get\_length():
\begin{itemize}
\item output: $out := length$
\item exception: none
\end{itemize}

\noindent get\_head():
\begin{itemize}
\item output: $out := head$
\item exception: none
\end{itemize}

\noindent get\_tail():
\begin{itemize}
\item output: $out := tail$
\item exception: none
\end{itemize}

\newpage


\section* {Point ADT Module}

\subsection*{Template Module}

PointT

\subsection* {Uses}

N/A

\subsection* {Syntax}

\subsubsection* {Exported Types}

PointT = ?

\subsubsection* {Exported Constants}

MAX\_X = 10
MAX\_Y = 10

\subsubsection* {Exported Access Programs}

\begin{tabular}{| l | l | l | l |}
\hline
\textbf{Routine name} & \textbf{In} & \textbf{Out} & \textbf{Exceptions}\\
\hline
PointT & real, real & PointT & InvalidPointException\\
\hline
xcrd & ~ & real & ~\\
\hline
ycrd & ~ & real & ~\\
\hline
dist & PointT & real & ~\\
\hline
\end{tabular}

\subsection* {Semantics}

\subsubsection* {State Variables}

$xc$: real\\
$yc$: real

\subsubsection* {State Invariant}

none

\subsubsection* {Assumptions}
The constructor PointT is called for each abstract object before any other access routine is called for that
object.  The constructor cannot be called on an existing object.

\subsubsection* {Access Routine Semantics}

PointT($x, y$):
\begin{itemize}
\item transition: $xc, yc := x, y$
\item output: $out := \mathit{self}$
\item exception
 $exc := ((\neg(0 \leq x \leq \mbox{MAX\_X}) \vee \neg(0 \leq y \leq \mbox{MAX\_Y})) \Rightarrow
\mbox{InvalidPointException})$
\end{itemize}

\noindent xcrd():
\begin{itemize}
\item output: $out := xc$
\item exception: none
\end{itemize}

\noindent ycrd():
\begin{itemize}
\item output: $out := yc$
\item exception: none
\end{itemize}

\noindent dist($p$):
\begin{itemize}
\item output: $out := \sqrt{(\mathit{self}.xc - p.xc)^2 + (\mathit{self}.yc - p.yc)^2}$
\item exception: none
\end{itemize}

\newpage


\end {document}
