\documentclass[12pt]{article}

\usepackage{graphicx}
\usepackage{paralist}
\usepackage{amsfonts}
\usepackage{listings}
\usepackage{hyperref}
\usepackage{indentfirst}
\hypersetup{colorlinks=true,
    linkcolor=blue,
    citecolor=blue,
    filecolor=blue,
    urlcolor=blue,
    unicode=false}

\oddsidemargin 0mm
\evensidemargin 0mm
\textwidth 160mm
\textheight 200mm

\pagestyle {plain}
\pagenumbering{arabic}

\newcounter{stepnum}

\title{Assignment 3 - 2AA4}
\author{Mohamed Bengezi bengezim}

\begin {document}

\maketitle
The contents of this report include the specifications of RegionT and PathCalculation modules, as
well as a critique of the interface for the modules for Part 1 of Assignment 3.



\section* {Region Module}

\subsection* {Template Module}

RegionT

\subsection* {Uses}

PointT, Constants

\subsection* {Syntax}

\subsubsection* {Exported Types}

RegionT = ?

\subsubsection* {Exported Access Programs}

\begin{tabular}{| l | l | l | l |}
\hline
\textbf{Routine name} & \textbf{In} & \textbf{Out} & \textbf{Exceptions}\\
\hline
RegionT & PointT, real, real & RegionT & InvalidRegionException\\
\hline
pointInRegion & PointT & boolean & ~\\
\hline 
\end{tabular}

\subsection* {Semantics}

\subsubsection* {State Variables}

$\mathit{lower\_left}$: PointT {\it //coordinates of the lower left corner of the region}\\
$\mathit{width}$: real {\it //width of the rectangular region}\\
$\mathit{height}$: real {\it //height of the rectangular region}

\subsubsection* {State Invariant}
None

\subsubsection* {Assumptions}
The RegionT constructor is called for each abstract object before any other access routine is called for that
object.  The constructor can only be called once.

\subsubsection* {Access Routine Semantics}

\noindent RegionT($p, w, h$):
\begin{itemize}
\item transition: $\mathit{lower\_left}, \mathit{width}, \mathit{height} := p, w, h$
\item output: $out := \mathit{self}$
\item exception: $exc := (\lnot (0 \leq p.xcrd() \leq \mbox{Contants.MAX\_X} - w) \vee (0 \leq p.ycrd() \leq \mbox{Constants.MAX\_Y} - h)) \Rightarrow
\mbox{InvalidRegionException})$
\end{itemize}

\noindent pointInRegion($p$):
\begin{itemize}
\item output: $\mathit{out} :=  \exists (p, q : PointT| q.ycrd() \in [lower\_left.ycrd() .. height] \land q.xcrd() \in [lower\_left.xcrd() .. width] : p.dist(q) < TOLERANCE) $
\item exception: none
\end{itemize}

\newpage

\section* {Path Calculation Module}

\subsection* {Module}

PathCalculation

\subsection* {Uses}

Constants, PointT, RegionT, PathT, Obstacles, Destinations, SafeZone, Map

\subsection* {Syntax}

\subsubsection* {Exported Access Programs}

\begin{tabular}{| l | l | l | l |}
\hline
\textbf{Routine name} & \textbf{In} & \textbf{Out} & \textbf{Exceptions}\\
\hline
is\_validSegment & PointT, PointT & boolean & ~\\
\hline
is\_validPath & PathT & boolean & ~\\
\hline
is\_shortestPath & PathT & boolean & ~\\
\hline
totalDistance & PathT & real & ~\\
\hline
totalTurns & PathT & integer & ~\\
\hline
estimatedTime & PathT & real & ~\\
\hline
%sortPathList & PathListT & PathListT & ~\\
%\hline

\end{tabular}

\subsection* {Semantics}

\subsubsection* {State Invariant}
None

\subsubsection* {Assumptions}
The RegionT constructor is called for each abstract object before any other access routine is called for that
object.  The constructor can only be called once.

\subsubsection* {Access Routine Semantics}

\noindent is\_validSegment($p_1, p_2$):
\begin{itemize}
\item output: $\mathit{out} := \forall( p, q : PointT | (q.dist(p_1) + q.dist(p_2) = p_1.dist(p_2)) \land (p.dist(q) \leq \mbox{Constants.TOLERANCE}) : \forall (i : \mathbb{N} | i \in [0..|Obs|-1] : \lnot ((Obstacles.getval(i)).pointInRegion(p))))$
\item exception: None
\end{itemize}

\noindent is\_validPath($p$):
\begin{itemize}
\item output: $out :=  \exists(q : PointT \land i : \mathbb{N} | (q.dist(p.getval(i)) + q.dist(p.getval(i+1))\\
 = p.getval(i).dist(Path.getval(i+1))) \forall(i,j,k : \mathbb{N} \land s : SafeZone \land  d : Destinations| i \in [0..s.size()-1]
\land Maps.get\_safeZone.getval(i).pointInRegion(p.getval(0)) \\
\land Maps.get\_safeZone.getval(i).pointInRegion(p.getval(p.size()-1))
\land \\
j \in  [0..Maps.get\_destinations.size()-1] \land k \in [1..p.size()-1] \land \\
Maps.get\_destinations.getval(j).pointInRegion(p.getval(0))  \land Maps.get\\ \_destinations.getval(j).pointInRegion(p.getval(k)): p.isvalidSegment(p.getval(k-1), p.getval(k))))$
\item exception: None
\end{itemize}


\noindent is\_shortestPath($p$):
\begin{itemize}
\item output: $\mathit{out} :=  \forall (x : PathT | PathCalculation.is\_validPath(x) : PathCalculation.totalDistance(p) \\
\leq  PathCalculation.totalDistance(x)) $
\item exception: none
\end{itemize}

\noindent totalDistance($p$):
\begin{itemize}
\item output: $out := + ( i: \mathbb{N}  |  i \in [0 ..  p.size()-1] \land  i < p.size() - 2 :  p.getval(i).dist(p.getval(i+1)) )$
\item exception: None
\end{itemize}


\noindent totalTurns($p$):
\begin{itemize}
\item output: $\mathit{out} := + ( i: \mathbb{N}  |  i \in [0 ..  |p|-3] \land (p_.getVal([i+1])_.dist(p_.getVal([i])) + p_.getVal([i+2])_.dist(p_.getVal([i+1]))) \neq p_.getVal([i+2])_.dist(p_.getVal([i])) : 1 )$
\item exception: none
\end{itemize}

\noindent estimatedTime($p$):
\begin{itemize}
\item output: $\mathit{out} := (Constants.VECLOCITY\_LINEAR \times totalDistance(p)) + ( + ( i: \mathbb{N} | i \in [0 .. |p|-3] : Constants.VELOCITY\_ANGULAR \times pathAngle(p_.getVal(i), p_.getVal(i+1), p_.getVal(i+2))))$
\item exception: none
\end{itemize}
    
\subsubsection*{Local Functions}
angle: PointT $\times$ PointT  $\times$ PointT  $\rightarrow$  real
\noindent pathAngle($p, q, r$):
\begin{itemize}
\item output: $\mathit{out} := \arccos (((r_.ycrd() - q_.ycrd()) \times (q_ycrd() - p_.ycrd()) + ((r_.xcrd() - q_.xcrd()) \times (q_.xcrd() - p_.xcrd())) / (r_.dist(q) \times q_.dist(p)))$
\item exception: none
\end{itemize}

\newpage

\section* {Point ADT Module}

\subsection*{Template Module}

PointT

\subsection* {Uses}



\subsection* {Syntax}

\subsubsection* {Exported Types}

PointT = ?

\subsubsection* {Exported Constants}

MAX\_X = 10
MAX\_Y = 10

\subsubsection* {Exported Access Programs}

\begin{tabular}{| l | l | l | l |}
\hline
\textbf{Routine name} & \textbf{In} & \textbf{Out} & \textbf{Exceptions}\\
\hline
PointT & real, real & PointT & InvalidPointException\\
\hline
xcrd & ~ & real & ~\\
\hline
ycrd & ~ & real & ~\\
\hline
dist & PointT & real & ~\\
\hline
\end{tabular}

\subsection* {Semantics}

\subsubsection* {State Variables}

$xc$: real\\
$yc$: real

\subsubsection* {State Invariant}

none

\subsubsection* {Assumptions}
The constructor PointT is called for each abstract object before any other access routine is called for that
object.  The constructor cannot be called on an existing object.

\subsubsection* {Access Routine Semantics}

PointT($x, y$):
\begin{itemize}
\item transition: $xc, yc := x, y$
\item output: $out := \mathit{self}$
\item exception
 $exc := ((\neg(0 \leq x \leq \mbox{MAX\_X}) \vee \neg(0 \leq y \leq \mbox{MAX\_Y})) \Rightarrow
\mbox{InvalidPointException})$
\end{itemize}

\noindent xcrd():
\begin{itemize}
\item output: $out := xc$
\item exception: none
\end{itemize}

\noindent ycrd():
\begin{itemize}
\item output: $out := yc$
\item exception: none
\end{itemize}

\noindent dist($p$):
\begin{itemize}
\item output: $out := \sqrt{(\mathit{self}.xc - p.xc)^2 + (\mathit{self}.yc - p.yc)^2}$
\item exception: none
\end{itemize}

\newpage
 
\end {document}
